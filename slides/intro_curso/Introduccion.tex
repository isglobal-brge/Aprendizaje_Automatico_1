% Options for packages loaded elsewhere
\PassOptionsToPackage{unicode}{hyperref}
\PassOptionsToPackage{hyphens}{url}
%
\documentclass[
  ignorenonframetext,
]{beamer}
\usepackage{pgfpages}
\setbeamertemplate{caption}[numbered]
\setbeamertemplate{caption label separator}{: }
\setbeamercolor{caption name}{fg=normal text.fg}
\beamertemplatenavigationsymbolsempty
% Prevent slide breaks in the middle of a paragraph
\widowpenalties 1 10000
\raggedbottom
\setbeamertemplate{part page}{
  \centering
  \begin{beamercolorbox}[sep=16pt,center]{part title}
    \usebeamerfont{part title}\insertpart\par
  \end{beamercolorbox}
}
\setbeamertemplate{section page}{
  \centering
  \begin{beamercolorbox}[sep=12pt,center]{part title}
    \usebeamerfont{section title}\insertsection\par
  \end{beamercolorbox}
}
\setbeamertemplate{subsection page}{
  \centering
  \begin{beamercolorbox}[sep=8pt,center]{part title}
    \usebeamerfont{subsection title}\insertsubsection\par
  \end{beamercolorbox}
}
\AtBeginPart{
  \frame{\partpage}
}
\AtBeginSection{
  \ifbibliography
  \else
    \frame{\sectionpage}
  \fi
}
\AtBeginSubsection{
  \frame{\subsectionpage}
}
\usepackage{amsmath,amssymb}
\usepackage{lmodern}
\usepackage{ifxetex,ifluatex}
\ifnum 0\ifxetex 1\fi\ifluatex 1\fi=0 % if pdftex
  \usepackage[T1]{fontenc}
  \usepackage[utf8]{inputenc}
  \usepackage{textcomp} % provide euro and other symbols
\else % if luatex or xetex
  \usepackage{unicode-math}
  \defaultfontfeatures{Scale=MatchLowercase}
  \defaultfontfeatures[\rmfamily]{Ligatures=TeX,Scale=1}
\fi
% Use upquote if available, for straight quotes in verbatim environments
\IfFileExists{upquote.sty}{\usepackage{upquote}}{}
\IfFileExists{microtype.sty}{% use microtype if available
  \usepackage[]{microtype}
  \UseMicrotypeSet[protrusion]{basicmath} % disable protrusion for tt fonts
}{}
\makeatletter
\@ifundefined{KOMAClassName}{% if non-KOMA class
  \IfFileExists{parskip.sty}{%
    \usepackage{parskip}
  }{% else
    \setlength{\parindent}{0pt}
    \setlength{\parskip}{6pt plus 2pt minus 1pt}}
}{% if KOMA class
  \KOMAoptions{parskip=half}}
\makeatother
\usepackage{xcolor}
\IfFileExists{xurl.sty}{\usepackage{xurl}}{} % add URL line breaks if available
\IfFileExists{bookmark.sty}{\usepackage{bookmark}}{\usepackage{hyperref}}
\hypersetup{
  pdftitle={Introducción al curso},
  hidelinks,
  pdfcreator={LaTeX via pandoc}}
\urlstyle{same} % disable monospaced font for URLs
\newif\ifbibliography
\usepackage{color}
\usepackage{fancyvrb}
\newcommand{\VerbBar}{|}
\newcommand{\VERB}{\Verb[commandchars=\\\{\}]}
\DefineVerbatimEnvironment{Highlighting}{Verbatim}{commandchars=\\\{\}}
% Add ',fontsize=\small' for more characters per line
\usepackage{framed}
\definecolor{shadecolor}{RGB}{248,248,248}
\newenvironment{Shaded}{\begin{snugshade}}{\end{snugshade}}
\newcommand{\AlertTok}[1]{\textcolor[rgb]{0.94,0.16,0.16}{#1}}
\newcommand{\AnnotationTok}[1]{\textcolor[rgb]{0.56,0.35,0.01}{\textbf{\textit{#1}}}}
\newcommand{\AttributeTok}[1]{\textcolor[rgb]{0.77,0.63,0.00}{#1}}
\newcommand{\BaseNTok}[1]{\textcolor[rgb]{0.00,0.00,0.81}{#1}}
\newcommand{\BuiltInTok}[1]{#1}
\newcommand{\CharTok}[1]{\textcolor[rgb]{0.31,0.60,0.02}{#1}}
\newcommand{\CommentTok}[1]{\textcolor[rgb]{0.56,0.35,0.01}{\textit{#1}}}
\newcommand{\CommentVarTok}[1]{\textcolor[rgb]{0.56,0.35,0.01}{\textbf{\textit{#1}}}}
\newcommand{\ConstantTok}[1]{\textcolor[rgb]{0.00,0.00,0.00}{#1}}
\newcommand{\ControlFlowTok}[1]{\textcolor[rgb]{0.13,0.29,0.53}{\textbf{#1}}}
\newcommand{\DataTypeTok}[1]{\textcolor[rgb]{0.13,0.29,0.53}{#1}}
\newcommand{\DecValTok}[1]{\textcolor[rgb]{0.00,0.00,0.81}{#1}}
\newcommand{\DocumentationTok}[1]{\textcolor[rgb]{0.56,0.35,0.01}{\textbf{\textit{#1}}}}
\newcommand{\ErrorTok}[1]{\textcolor[rgb]{0.64,0.00,0.00}{\textbf{#1}}}
\newcommand{\ExtensionTok}[1]{#1}
\newcommand{\FloatTok}[1]{\textcolor[rgb]{0.00,0.00,0.81}{#1}}
\newcommand{\FunctionTok}[1]{\textcolor[rgb]{0.00,0.00,0.00}{#1}}
\newcommand{\ImportTok}[1]{#1}
\newcommand{\InformationTok}[1]{\textcolor[rgb]{0.56,0.35,0.01}{\textbf{\textit{#1}}}}
\newcommand{\KeywordTok}[1]{\textcolor[rgb]{0.13,0.29,0.53}{\textbf{#1}}}
\newcommand{\NormalTok}[1]{#1}
\newcommand{\OperatorTok}[1]{\textcolor[rgb]{0.81,0.36,0.00}{\textbf{#1}}}
\newcommand{\OtherTok}[1]{\textcolor[rgb]{0.56,0.35,0.01}{#1}}
\newcommand{\PreprocessorTok}[1]{\textcolor[rgb]{0.56,0.35,0.01}{\textit{#1}}}
\newcommand{\RegionMarkerTok}[1]{#1}
\newcommand{\SpecialCharTok}[1]{\textcolor[rgb]{0.00,0.00,0.00}{#1}}
\newcommand{\SpecialStringTok}[1]{\textcolor[rgb]{0.31,0.60,0.02}{#1}}
\newcommand{\StringTok}[1]{\textcolor[rgb]{0.31,0.60,0.02}{#1}}
\newcommand{\VariableTok}[1]{\textcolor[rgb]{0.00,0.00,0.00}{#1}}
\newcommand{\VerbatimStringTok}[1]{\textcolor[rgb]{0.31,0.60,0.02}{#1}}
\newcommand{\WarningTok}[1]{\textcolor[rgb]{0.56,0.35,0.01}{\textbf{\textit{#1}}}}
\usepackage{graphicx}
\makeatletter
\def\maxwidth{\ifdim\Gin@nat@width>\linewidth\linewidth\else\Gin@nat@width\fi}
\def\maxheight{\ifdim\Gin@nat@height>\textheight\textheight\else\Gin@nat@height\fi}
\makeatother
% Scale images if necessary, so that they will not overflow the page
% margins by default, and it is still possible to overwrite the defaults
% using explicit options in \includegraphics[width, height, ...]{}
\setkeys{Gin}{width=\maxwidth,height=\maxheight,keepaspectratio}
% Set default figure placement to htbp
\makeatletter
\def\fps@figure{htbp}
\makeatother
\setlength{\emergencystretch}{3em} % prevent overfull lines
\providecommand{\tightlist}{%
  \setlength{\itemsep}{0pt}\setlength{\parskip}{0pt}}
\setcounter{secnumdepth}{-\maxdimen} % remove section numbering
%% change fontsize of R code
\let\oldShaded\Shaded
\let\endoldShaded\endShaded
\renewenvironment{Shaded}{\fontsize{10}{9}\oldShaded}{\endoldShaded}

%% change fontsize of output
\let\oldverbatim\verbatim
\let\endoldverbatim\endverbatim
%\renewenvironment{verbatim}{\tiny\oldverbatim}{\endoldverbatim}
\renewenvironment{verbatim}{\fontsize{6.5}{7.5}\selectfont\oldverbatim}{\endoldverbatim}

%% pages
\setbeamertemplate{navigation symbols}{} 
\setbeamertemplate{footline}[frame number]
\usepackage{enumerate} 
\usepackage{multicol} 
\usepackage{lipsum}
\usepackage{multirow}
\ifluatex
  \usepackage{selnolig}  % disable illegal ligatures
\fi

\title{Introducción al curso}
\subtitle{Aprendizaje automático 1}
\author{Juan R Gonzalez\\
\href{mailto:juanr.gonzalez@isglobal.org}{\nolinkurl{juanr.gonzalez@isglobal.org}}}
\date{}
\institute{UAB - Department of Mathematics\\
BRGE - Bioinformatics Research Group in Epidemiology\\
ISGlobal - Barcelona Institute for Global Health\\
\url{http://brge.isglobal.org}}

\begin{document}
\frame{\titlepage}

\begin{frame}{Introducción}
\protect\hypertarget{introducciuxf3n}{}
\includegraphics{fig/intro_AA}

Sobre 2010 DL obtuvo una gran popularidad ya que ha permitido acercarse
a sitemas de inteligencia artificial de forma más eficientge que ML. Los
tres términos están ligados y cada uno forma una parte esencial de los
otros. DL permite llevar a cabo ML, que en última instancia permite la
AI.

No obstante, es más fácil aprender ML como herramienta para AI. En el
siguente curso (AA\_2) veréis cómo llevar a cabo técnicas de DL.
\end{frame}

\begin{frame}{Introducción}
\protect\hypertarget{introducciuxf3n-1}{}
\includegraphics{fig/ML_DL}
\end{frame}

\begin{frame}{Ejemplos}
\protect\hypertarget{ejemplos}{}
\includegraphics[width=0.5\textwidth,height=\textheight]{fig/ojos_rojos}

\includegraphics{fig/deep_learning}
\end{frame}

\begin{frame}{Ejemplos}
\protect\hypertarget{ejemplos-1}{}
\includegraphics{fig/ML}
\end{frame}

\begin{frame}[fragile]{Temario}
\protect\hypertarget{temario}{}
\begin{itemize}
\tightlist
\item
  Introducción a Tidyverse
\item
  Introducción al aprendizaje automático
\item
  Regresión lineal y logística (nomogramas)
\item
  Dealing with Big Data in R (MapReduce)
\item
  La librería \texttt{caret}
\item
  Pasos previos a la creación de un modelo predictivo y medidas de
  validación
\item
  Métodos de aprendizaje automático

  \begin{itemize}
  \tightlist
  \item
    K-vecinos más cercanos (KNN)
  \item
    Análisis discriminante lineal (LDA)
  \item
    Máquinas de soporte vectorial (SVM)
  \item
    Árboles de decisión (clasificación, regresión, bagged trees, random
    forest)
  \item
    Boosting (AdaBoost, GBM básico y estocástico, XGBoost)
  \end{itemize}
\item
  Respuesta no balanceada
\end{itemize}
\end{frame}

\begin{frame}{Logística del curso}
\protect\hypertarget{loguxedstica-del-curso}{}
\begin{itemize}
\item
  Clases lunes de 15:00 a 17:00 (teoría) + 17:00 a 19:00 (prácticas)
\item
  Clases presenciales aula C3/028
\item
  Clases no presenciales (bookdown + algunas grabadas). Tendrá que
  hacerse las preguntas de autoevaluación el mismo día antes de las
  21:00 horas (no negociable)
\item
  Cada semana habrá una práctica similar a lo que se ha explicado en
  clase, pero con otros datos reales. También habrán ``Data analysis
  Challenges''
\item
  Cada semana se pondrá una noticia con lo que se va a hacer cada lunes.
\end{itemize}
\end{frame}

\begin{frame}{Material del curso}
\protect\hypertarget{material-del-curso}{}
\begin{itemize}
\item
  No seguiremos ningún libro de texto porque estamos tratando muchos
  temas que están muy bien explicados en varios libros y, sobre todo, en
  materiales públicos. He creado un bookdown accesible en
  \url{https://isglobal-brge.github.io/Aprendizaje_Automatico_1/} que
  describe el contenido del curso
\item
  Habrán presentaciones en diapositivas que estarán disponibles en el
  Moodle de la asignatura
\item
  Habrán link a material opcional
\item
  Las actividades en el Moodle estarán clasificadas como:

  \begin{itemize}
  \tightlist
  \item
    P: puntuable (se evaluará según el contenido)
  \item
    P2: puntuable (no se evaluará si está bien - se dará la solución: 10
    entregado, 5 entregado no completo, 0 no entregado)\\
  \item
    I: información
  \item
    O: opcional
  \end{itemize}
\end{itemize}
\end{frame}

\begin{frame}{Metodología}
\protect\hypertarget{metodologuxeda}{}
\begin{itemize}
\item
  Clase de teoria: se definen y se explican los diferentes métodos con
  sus características particulares y se muestran ejemplos concretos.
\item
  Clase de prácticas: se trabajan los métodos explicados en clase de
  teoria con diversos conjuntos de datos utilizando el lenguaje de
  programación R.
\end{itemize}

Se considera que, para cada hora de teoria y prácticas, el alumno deberá
dedicar una hora adicional para la preparación y/o finalización de la
sesión.

\begin{itemize}
\tightlist
\item
  Preguntas de auto-evaluación que se llevarán a cabo en el Moodle que
  servirá para consolidar los conocimientos aprendidos en las sesiones
  teóricas y que en las sesiones no presenciales se tendrá que hacer el
  mismo día antes de las 21:00.
\end{itemize}
\end{frame}

\begin{frame}{Evaluación}
\protect\hypertarget{evaluaciuxf3n}{}
La nota final se basará en las siguientes notas ponderadas

\begin{itemize}
\tightlist
\item
  50\% Nota Examen final (tipo test - conceptos)
\item
  40\% Nota Prácticas (compuesta por prácticas semanales más una
  práctica final)
\item
  10\% Nota preguntas de Auto-Evaluación
\end{itemize}

Se requerirá tener un 5 en el Examen final para aprobar la asignatura,
en caso contrario el alumno deberá presentarse al examen de
recuperación. Existe la posibilidad de aprobar el examen con un 4.5 si
el alumno ha participado de forma activa en el foro de la asignatura.
\end{frame}

\begin{frame}{Fechas de Evaluación}
\protect\hypertarget{fechas-de-evaluaciuxf3n}{}
\begin{itemize}
\tightlist
\item
  Examen parcial (no hay)
\item
  Examen final (fecha por saber)
\item
  Recuperación Examen 70\% (fecha por saber) + Notas de prácticas 30\%
\end{itemize}
\end{frame}

\begin{frame}[fragile]{Session info}
\protect\hypertarget{session-info}{}
\begin{Shaded}
\begin{Highlighting}[]
\FunctionTok{sessionInfo}\NormalTok{()}
\end{Highlighting}
\end{Shaded}

\begin{verbatim}
R version 4.0.2 (2020-06-22)
Platform: x86_64-w64-mingw32/x64 (64-bit)
Running under: Windows 10 x64 (build 18362)

Matrix products: default

locale:
[1] LC_COLLATE=Spanish_Spain.1252  LC_CTYPE=Spanish_Spain.1252   
[3] LC_MONETARY=Spanish_Spain.1252 LC_NUMERIC=C                  
[5] LC_TIME=Spanish_Spain.1252    

attached base packages:
[1] stats     graphics  grDevices utils     datasets  methods   base     

loaded via a namespace (and not attached):
 [1] compiler_4.0.2  magrittr_1.5    tools_4.0.2     htmltools_0.5.0
 [5] yaml_2.2.1      stringi_1.4.6   rmarkdown_2.3   knitr_1.29     
 [9] stringr_1.4.0   xfun_0.16       digest_0.6.25   rlang_0.4.7    
[13] evaluate_0.14  
\end{verbatim}
\end{frame}

\end{document}
